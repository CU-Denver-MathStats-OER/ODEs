%UNIT 16: Introduction to Laplace Transforms
%%%%%%%%%%%%%%%%%%%%%%%%%%%
%%%% Put the following at the top of each .tex file  %
\pagestyle{fancy}
\renewcommand{\theUnit}{Sections 7.4 and 7.5}
\ifthenelse{\isundefined{\UnitPageNumbers}}{}{\setcounter{page}{1}}
\rhead{\theUnit: Inverse Laplace Transforms and Diff. Eq.}
\lhead{\includegraphics[width=1.25cm]{IODE-logo.png}}
\rfoot{\mypage}
\lfoot{}
\cfoot{}
\fancypagestyle{firstfooter}{\footskip = 50pt}
\renewcommand{\footrulewidth}{.4pt}
%%%%%%%%%%%%%%%%%%%%%%%%%%%
\vspace*{-20pt} \thispagestyle{firstfooter}

\pagebegin{Common Laplace Transforms and Properties}

\begin{center}
\begin{tabular}{|l|l|}
\hline
$f(t)$ & $F(s) = \Lap \left\{ f(t) \right\}$ \\
\hline
$1$ & $\dsty \frac{1}{s}, \ s > 0$\\
\hline
$e^{at}$ & $\dsty \frac{1}{s-a}, \ s > a$\\
\hline
$t^n, \ n=1,2, \ldots$ & $\dsty \frac{n!}{s^{n+1}}, \ s > 0$\\
\hline
$\sin{(bt)}$ & $\dsty \frac{b}{s^2+b^2}, \ s > 0$\\
\hline
$\cos{(bt)}$ & $\dsty \frac{s}{s^2+b^2}, \ s > 0$\\
\hline
$e^{at}t^n, \ n=1,2, \ldots$ & $\dsty \frac{n!}{(s-a)^{n+1}}, \ s > 0$\\
\hline
$e^{at}\sin{(bt)}$ & $\dsty \frac{b}{(s-a)^2+b^2}, \ s > a$\\
\hline
$e^{at}\cos{(bt)}$ & $\dsty \frac{s-a}{(s-a)^2+b^2}, \ s > a$\\
\hline
\end{tabular}
\end{center}

 \textbf{Properties:}

\bb
\ii $\mathscr{L} \left\{ cf(t) \right\} = c  \mathscr{L} \left\{ f(t) \right\}$, where $c$ is a constant.
\ii $\mathscr{L} \left\{ f_1(t) + f_2(t) \right\} = \mathscr{L} \left\{ f_1(t) \right\} + \mathscr{L} \left\{ f_2(t)\right\}$
\ii If $F(s) = \mathscr{L} \left\{ f(t) \right\}$ exists for all $s > \alpha$, then $\dsty \mathscr{L} \left\{ e^{at} f(t) \right\} = F(s-a)$ for all $s>\alpha + a$. 
\ii If $F(s) =\mathscr{L} \left\{ f(t) \right\}$ exists for all $s > \alpha$, then for all $s>\alpha$:
\bi
\ii $\dsty \mathscr{L} \left\{ f'(t) \right\} = sF(s) -f(0)$ for all $s > \alpha$, and thus
\ii $\dsty \mathscr{L} \left\{ f^{(n)}(t) \right\} = s^n\mathscr{L} \{ f(t) \}-s^{n-1} f(0)- s^{n-2} f'(0) - \ldots - f^{(n-1)}(0).$
\ei
\ii If $F(s) =\mathscr{L} \left\{ f(t) \right\}$ exists for all $s > \alpha$, then $\dsty \mathscr{L} \left\{ t^n f(t) \right\} = (-1)^n \frac{d^nF}{ds^n}$ for all $s > \alpha$.
\ee

 \textbf{Explain in Words:}

\bb
\ii Describe properties 3, 4, and 5 in words. For example in property 3, multiplying $f(t)$ by $e^{at}$ and then taking the Laplace transform has what affect on $\mathscr{L} \left\{ f(t) \right\}$? \label{17problem1}
\ee

\clearpage

\pagebegin{Solving Diff. Eqs. with Inverse Laplace Transforms }

\bb[resume]
\ii Solve $y''-y=-t$ with $y(0)=0$ and $y'(0)=1$.
\bb
\ii Using the properties, apply the Laplace transform to both sides: \label{17problem2parta}
\[ \Lap \{ y'' -y \} = \Lap \{ -t \} .\]
\vfill
\ii Using your answer in \ref{17problem2parta}, solve for $\Lap \{ y(t) \}=Y(s)$.  \label{17problem2partb}
\vfill
\ii Use the table of common Laplace transforms to identify what function $y(t)$ has $\Lap \{ y(t) \}=Y(s)$. \label{17problem2partc}
\vfill
\ee
\ee

In \ref{17problem2partc}, we are apply the \textbf{Inverse Laplace Transform} to $Y(s)$ in order to identify $y(t) = \Lap^{-1} \{ Y(s) \}$.

\clearpage

\pagebegin{Section 7.4: Inverse Laplace Transforms}

Given $F(s)$, if there is a function $f(t)$ that is continuous on $\lbrack 0 , \infty )$ and satisfies $\Lap \{ f \} = F(s)$, then
we say $f(t)$ is the \textbf{inverse Laplace transform} of $F(s)$ which is denoted by
\[ \mathbf{f(t) = \Lap^{-1} \{ F(s) \}} .\]

\bb[resume]
\ii Determine whether the inverse Laplace transform is of the form $t^n$, $\cos{(bt)}$, $\sin{(bt)}$, or $e^{at}$.  \label{17problem3}

\bb
\ii $\dsty F(s) = \frac{1}{s^2}$  \label{17problem3a} %, then  $f(t) = \ILap \left\{ \frac{1}{s^2} \right\} = t$.
\vfill
\ii $\dsty F(s) = \frac{2}{s^2+4}$  \label{17problem3b}%, then  $f(t) = \ILap \left\{ \frac{2}{s^2+4} \right\} =  \sin{(2t)}$. 
\vfill
\ii $\dsty F(s) = \frac{4s}{s^2+9}$  \label{17problem3c}%, then  $f(t) = \ILap \left\{ \frac{4s}{s^2+9} \right\} =  4\cos{(3t)}$.
\vfill
\ii $\dsty F(s) = \frac{2}{s+6}$  \label{17problem3d}%, then  $f(t) = \ILap \left\{ \frac{2}{s+6} \right\} =  2e^{-6t}$.
\vfill
\ee

\ii Find the inverse Laplace transform of $\dsty F(s) = \frac{s+2}{s^2+4s+11}$ by answering the questions below.  \label{17problem4}

\bb
\ii Complete the square for the expression in the denominator of $F(s)$ to express \newline $s^2+4s+11=(s-a)^2+b$. \label{17problem4a}
\vfill
\ii Use the table of common Laplace transforms to identify $\Lap^{-1} \{ F(s)\}$. \label{17problem4b}
\vfill
\ee

\clearpage

\ii Find the inverse Laplace transform of the function.\label{17problem5}

\bb
\ii $\dsty F(s) = \frac{5s-10}{s^2-3s-4}$ \label{17problem5a}
\vfill
\ii $\dsty F(s) = \frac{3s-15}{2s^2-4s+10}$ \label{17problem5b}
\vfill
\ii $\dsty F(s) = \frac{-5s-36}{(s+2)(s^2+9)}$ \label{17problem5c}
\vfill
\ee

\clearpage
\pagebegin{Section 7.5: Solving Initial Value Problems}

\ii Solve the initial value problem using Laplace Transforms (not previous methods).  \label{17problem6}

\bb
\ii $y''-2y'+5y=0$ with $y(0)=2$ and $y'(0)=4$. \label{17problem6a}
\vfill
\ii $y''-y'-2y=0$ with $y(0)=-2$ and $y'(0)=5$. \label{17problem6b} %Q2
\vfill
\clearpage
\ii $y''-4y'+5y=4e^{3t}$ with $y(0)=2$ and $y'(0)=7$. \label{17problem6c} %Q6
\vfill
\ii $ty''-ty'+y=2$ with $y(0)=2$ and $y'(0)=-1$. \label{17problem6d}
\vfill
\clearpage
\ii $y''+ty'-y=0$ with $y(0)=0$ and $y'(0)=3$. \label{17problem6e}
\vfill
\ee
\ee
