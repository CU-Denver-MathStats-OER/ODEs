%UNIT 3: AN ANALYTIC APPROACH
%%%%%%%%%%%%%%%%%%%%%%%%%%%
%%%% Put the following at the top of each .tex file  %
\pagestyle{fancy}
\renewcommand{\theUnit}{4.1-4.2}
\ifthenelse{\isundefined{\UnitPageNumbers}}{}{\setcounter{page}{1}}
\rhead{Application  \theUnit: Population and Mixture Models}
\lhead{\includegraphics[width=1.25cm]{IODE-logo.png}}
\rfoot{\mypage}
\lfoot{}
\cfoot{}
\fancypagestyle{firstfooter}{\footskip = 50pt}
\renewcommand{\footrulewidth}{.4pt}
%%%%%%%%%%%%%%%%%%%%%%%%%%%
\vspace*{-20pt} \thispagestyle{firstfooter}
\pagebegin{Mixture Problems (Application 4.2)}

\bbox
A one-compartment system consists of 

\bi
\ii $x(t)$ that represents the amount of a substance (such as salt) at time $t$. 
\ii an input rate of $x$. 
\ii an output rate of $x$. 
\ei

\[ \frac{dx}{dt} = \mbox{input rate} - \mbox{output rate} \]
\ebox

\bb
 \ii A brine solution of salt water that has concentration $0.05$ kg per L flows at a constant rate of 6 L per minute 
into a tank which is initially contains 50 L of a 1\% salt solution.   The brine solution flows out  of the tank at a rate of 4 L per minute. Let $x(t)$ denote the mass of the salt in the tank at time $t$ (in minutes). \textit{Note that 1\% salt solution means 1 kg of salt per 100 L of solution.}

\begin{enumerate}
\ii What is the input rate of $x$? \vspace{0.5in}
\ii What is the output rate of $x$?  \vspace{0.5in}
\ii What is the initial mass of the salt in the tank?  \vspace{0.5in}
\ii Construct an model for this initial value problem (but do not solve it).  \vspace{0.5in}
\ii What method(s) can we apply to solve the equation in (4) (but don't solve it)? 
\ee

\ee
%\bs \p 

%Now assume at time $t=0$, the tanks springs a leak so that the outflow of water is increasing linearly over time. 
%In particular, the outflow has increased from 6 L per min at time $t=0$ to 8 L per min at time $t=1$ minute. Adjust
%the differential equation to account for this leak. 

\clearpage

\pagebegin{Population Models (Section 4.1)}

\bbox
The \alert{Malthusian} law of population growth says the rate of change of the population, $\frac{dP}{dt}$,
is \alert{directly proportional to the population present}, $P$, at time $t$:
\[ \frac{dP}{dt} = kP, \quad P(0)=P_0.\]
\ebox

\bs 

\begin{ex}
 Let $P$ denote the population of the world (in billions) $t$ years since $1960$. In 1960 the world's population was approximately 3 billion, and the population growth is model by
\[ \frac{dP}{dt} =0.2P  \hspace{1in} , \ P(0)=3.\] 

Solving this model gives $P(t)=3e^{0.02t}$, and predicts the population in 2019 is $9.76$ billion.  \bs

\noindent \textbf{Why do you think predicted value is different from the actual value?}
\end{ex}

%In fact, this model predicts that in year 2635, the world's population will reach $1,800,000$ billion, meaning each person will have exactly 1 ft$^2$ of land to themself.

\clearpage

\pagebegin{The Logistic Models (Section 4.1)}

\bbox
We can construct our population model by considering::

\[ \frac{dP}{dt} = \bigg( \mbox{Birth Rate} \bigg) - \bigg( \mbox{Death Rate} \bigg).\] \bs

\bs

Competition within the population causes the populations to decrease (disease, murder, natural disasters, war, lack of food/water). If we assume the \alert{death rate is proportional to the total number of possible two-party interactions}, we get:

\[ \mbox{Death rate} = k_2 \left( \begin{array}{c} P\\ 2 \end{array} \right) = k_2 \left( \frac{P(P-1)}{2} \right) .\]

Note: $\dsty \left( \begin{array}{c} P\\ 2 \end{array} \right)$ denotes ``$P$ choose 2'', and in general we have
\[ \left( \begin{array}{c} n\\ k \end{array} \right) = \frac{n!}{k!(n-k)!}.\]
\ebox

\bs 

Taking both the birth and death rates into account, we get the \alert{Logistic model} for population change which we simplify:

\[ \frac{dP}{dt} = \bigg( \hspace{2in}  \bigg) - \bigg( \hspace{2in} \bigg) .\]  \bs

\bb[resume]
\ii Show that the model above can be rewritten in the form $\mathbf{\frac{dP}{dt} = -AP(P-L)}$ where $A$ and $L$ are positive constants.
\ee

\clearpage
\pagebegin{Practice: Population Model for Rabbits}

\bb[resume]
\ii A population of rabbits changes over time $t$ (in years) according to the logistic model
\[ \frac{dP}{dt} = 3P-\frac{1}{20}P^2 .\]

\bb
\ii For what initial population sizes $P_0$ will the population grow at first? \vspace{0.5in}
\ii For what initial population sizes $P_0$ will the population decrease at first? \vspace{0.5in}
\ii For what initial population sizes $P_0$ will the population never change? \vspace{0.5in}
\ii Explain, in practical terms, why answers in (a)-(c) makes sense. \vspace{1in}
\ii If the initial rabbit population is $P_0=P(0)=50$, find a solution to the 
initial value problem and find a formula for the population $P$ as a function of time $t$.
\ee
\ee

\pagebreak
\pagebegin{Practice: Chlorine Levels in Pool}

\bb[resume]
\ii A swimming pool whose volume is 10,000 gallons contains
water that is $0.01$\% chlorine. Starting at $t=0$, city water containing $0.001$\% chlorine
is pumped into the pool at a rate of 5 gal/min. The pool water flows out at the same rate. Let $x$
denote the amount of chlorine (in pounds) in the pool $t$ minutes since water has begun being
pumped into the pool.

\medskip

\textit{Note that a concentration of $0.01$\% chlorine solution means $0.01$ pounds of chlorine per 100 gallons of solution.}

\medskip

\bb
\ii Construct a differential equation for rate of change of the mass of chlorine (in pounds) $x$ in the pool at time $t$.  \vspace{1 in}
\ii Solve the initial value problem using the differential equation in (a) and the given initial \% concentration. \vfill
%\ii What is the percentage of chlorine in the pool after 1 hour? \vspace{1in}
\ii (Bonus) When will the pool water be $0.002$\% chlorine? 
\ee


\ee

