%UNIT 16: Introduction to Laplace Transforms
%%%%%%%%%%%%%%%%%%%%%%%%%%%
%%%% Put the following at the top of each .tex file  %
\pagestyle{fancy}
\renewcommand{\theUnit}{ 6.1}
\ifthenelse{\isundefined{\UnitPageNumbers}}{}{\setcounter{page}{1}}
\rhead{Section \theUnit: Introduction to Laplace Transforms}
\lhead{\includegraphics[width=1.25cm]{IODE-logo.png}}
\rfoot{\mypage}
\lfoot{}
\cfoot{}
\fancypagestyle{firstfooter}{\footskip = 50pt}
\renewcommand{\footrulewidth}{.4pt}
%%%%%%%%%%%%%%%%%%%%%%%%%%%
\vspace*{-20pt} \thispagestyle{firstfooter}
\pagebegin{Introduction to the  Laplace Transform}

One of the basic problem solving techniques in mathematics is to 
\bi
\ii transform a difficult problem into an easier one, 
\ii solve the easier problem, and 
\ii then use its solution to obtain a solution of the original problem.
\ei

\bigskip

For example. the reverse product rule (method of integrating factor) is used to transform a linear first order differential equation into an easier problem we can solve. In this chapter we study the method of \textbf{Laplace transforms}, which is one example of this technique. Like the method of integrating factors, Laplace transforms are \textbf{integral operators}. Solving by the method of Laplace transforms:
\bi
\ii Can be used to solve higher order linear differential equations.
\ii Can be applied for more complicated forcing functions.
\ii Requires initial conditions.
%\ii Has applications throughout physics and engineering.
\ei

\bs

\bbox
The \textbf{improper integral} of $g$ over $\lbrack a , \infty )$ is defined as
\[ \int_a^{\infty} g(t) \ dt = \lim_{N \to \infty} \int_a^N g(t) \ dt.\]
\bi
\ii We say the improper integral \textbf{converges} if the limit exists.
\ii Otherwise we say the improper integral \textbf{diverges}.
\ei
\ebox

\begin{enumerate}
\ii Determine whether $\dsty \int_0^{\infty} e^{-2t} \ dt$ converges or diverges.
\end{enumerate}

\clearpage
\pagebegin{Definition of the  Laplace Transform}
\bbox
Let $f(t)$ be a function on $\lbrack 0 , \infty )$. The \textbf{Laplace transform} of $f$ is the function $F$ defined by
\[ \Lap \left\{ f \right\} = F(s) = \int_0^{\infty} e^{-st}f(t) \ dt .\]
\bi
\ii The domain of $F(s)$ is all values of $s$ for which the integral converges.
\ii The functions $f$ and $F$ form a \textbf{transform pair}.
\ei
\ebox

\begin{enumerate}[resume]
  \ii Find and state the domain of the Laplace transform $F(s)=\Lap \left\{ f(t) \right\}$.
  \bb
\ii $f(t) = 2$, $t \geq 0$  \vfill
\ii $f(t) = t$ \vfill

\clearpage

\ii $f(t) = e^{3t}$ \vfill

\ii $g(t) = \cos{(bt)}$ where $b \ne 0$ is a constant.
\vfill

\clearpage

\ii $\dsty f(t) = \left\{ \begin{array}{ll} 
5 \ \ & 0 < t < 2 \\
e^{8t} \ \ & t >2 \end{array} \right.$ \vfill
\ee
\ee

\clearpage

\pagebegin{Common Laplace Transforms}

\begin{center}
\begin{tabular}{|l|l|}
\hline
 & \\
$f(t)$ & $F(s) = \Lap \left\{ f(t) \right\}$ \\
 & \\
\hline
 & \\
$f(t)=1$ & $\dsty F(s)=\frac{1}{s}, \ s > 0$\\
 & \\
\hline
 & \\
$f(t)=e^{at}$ & $\dsty F(s) = \frac{1}{s-a}, \ s > a$\\
 & \\
\hline
 & \\
$f(t)=t^n, \ n=1,2, \ldots$ & $\dsty F(s) = \frac{n!}{s^{n+1}}, \ s > 0$\\
 & \\
\hline
 & \\
$f(t)=\sin{(bt)}$ & $\dsty F(s) = \frac{b}{s^2+b^2}, \ s > 0$\\
 & \\
\hline
 & \\
$f(t)=\cos{(bt)}$ & $\dsty F(s) = \frac{s}{s^2+b^2}, \ s > 0$\\
 & \\
%\hline
%$e^{at}t^n, \ n=1,2, \ldots$ & $\dsty \frac{n!}{(s-a)^{n+1}}, \ s > 0$\\
%\hline
%$e^{at}\sin{(bt)}$ & $\dsty \frac{b}{(s-a)^2+b^2}, \ s > a$\\
%\hline
%$e^{at}\cos{(bt)}$ & $\dsty \frac{s-a}{(s-a)^2+b^2}, \ s > a$\\
\hline
\end{tabular}
\end{center}

